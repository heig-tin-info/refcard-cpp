\documentclass{article}

\usepackage[a4paper, landscape, margin=1cm]{geometry}
\usepackage{fontspec}
\usepackage[french]{babel}
\usepackage[fontsize=6.5pt]{scrextend}
\usepackage[T1]{fontenc}
\usepackage{multicol}
\usepackage{tabularx}
\usepackage{sectsty}
\usepackage{lmodern}
\usepackage{stix}
\usepackage{listings}
\usepackage[dvipsnames]{xcolor}
\usepackage{multirow}
\usepackage{titlesec}
\usepackage{graphicx}

\input{revision}

\ifdefined\tinted
  \pagecolor{ProcessBlue!30}
\fi

\setlength{\parskip}{0.2em}
\setlength{\parindent}{0em}

% Highlight configuration for C programming language
\lstset{
  language=c,
  breaklines=true,
  keywordstyle=\bfseries\color{black},
  basicstyle=\ttfamily\color{black},
  emphstyle={\em \color{gray}},
  emph={expr, type, NAME, ptr, name, expr, value, filename, label, member, type},
  mathescape=true,
  keepspaces=true,
  showspaces=false,
  showtabs=true,
  tabsize=3,
  columns=fullflexible,
  escapeinside={(*}{*)}
}

% Configuration
\renewcommand{\familydefault}{\sfdefault}

\sectionfont{\fontsize{12}{15}\selectfont}
\subsectionfont{\fontsize{10}{12}\selectfont}

\allsectionsfont{\sffamily\underline}

% No pages numbering
\pagenumbering{gobble}

% Titles and paragraphs more compact
\titlespacing*{\section}{0pt}{0pt}{0pt}
\titlespacing*{\subsection}{0pt}{0pt}{0pt}

\newlength\mybaselinestretch
\mybaselinestretch=0pt plus 0.02pt\relax
\addtolength{\baselineskip}{\mybaselinestretch}

\setlength\parindent{0pt}
\setlength\tabcolsep{1.5pt}
\setlength{\columnseprule}{0.4pt}

% Macros
\newcommand{\tab}{\hspace{2em}}
\newcommand{\etc}{\small \ldots}
\newcommand{\any}{$\hzigzag$~}
\newcommand{\spc}{$\mathvisiblespace$}
\newcommand{\cd}{\lstinline}

\begin{document}

\begin{multicols*}{3}

    \begin{tabularx}{\columnwidth}{lX}
        \raisebox{-\totalheight}{\includegraphics[width=1cm]{assets/heig-vd-black.pdf}} &
        \begin{center}
            {\Large \bf Carte de référence C++11/C++17} \\
            version \revision \ -- \revisiondate \\
        \end{center}
    \end{tabularx}

    Cette carte de référence peut être utilisée durant les travaux écrits et examens
    des cours \emph{progoo} et \emph{progoox} à moins que le contraire soit explicitement formulé.
    Elle est une liste non exhaustive des possibilités du langage C++ et un complément à la carte de référence du langage C.
    Ce travail est inspiré de \emph{Learn X in Y} et de la carte de référence de Matt Mahoney.
    Ci-dessous, la signification des termes utilisés dans cette carte de référence.

    \begin{tabularx}{\linewidth}{
            >{\hsize=0.5\hsize}X% 10% of 4\hsize
            >{\hsize=1.5\hsize}X% 30% of 4\hsize
            >{\hsize=0.5\hsize}X% 30% of 4\hsize
            >{\hsize=1.5\hsize}X% 30% of 4\hsize
            % sum=4.0\hsize for 4 columns
        }

        \tt \etc      & Continuation logique      & \tt \any  & N'importe quoi d'accepté     \\
        \tt /\any/    & Expression régulière      & \tt \spc  & Espace obligatoire           \\
        \cd{type}     & \tt int, long, float, ... & \cd{name} & \tt /[A-Za-z][A-Za-z0-9\_]+/ \\
        \cd{value}    & Valeur                    & \cd{NAME} & \tt /[A-Z][A-Z0-9\_]+/       \\
        \cd{filename} & Chemin de fichier relatif & \cd{expr} & e.g. \tt a + b               \\
    \end{tabularx}
    \hrule

    \section*{Différences C/C++}
    \begin{lstlisting}
sizeof('c') == sizeof(char) == 1 // Différence par rapport au C
void func();                     // func n'accepte aucun arguments
int *ip = nullptr;               // Utiliser à la place de NULL
#include <cstdio>                // Au lieu de <stdio.h> (pas de .h)
\end{lstlisting}

    \section*{Généralités et spécificités}

    \begin{lstlisting}
// Surcharge de fonctions
void say(char const* s) { printf("%s\n", s); }
void say(int u) { printf("%d\n", u); }
void main() { say("Hello"); say(15); }

// Arguments par défaut
void foo(int a = 1; int b = 4) { ... }
void main() {
   foo();      // a = 1, b = 4
   foo(20);    // a = 20, b = 4
   foo(20, 5); // a = 20, b = 5
}

for(auto x; iterable)   // Boucle itérative sur itérable
int& r = x;             // Référence r sur x
\end{lstlisting}

    \section*{Entrées sorties (streams)}

    \begin{lstlisting}
#include <iostream>
using namespace std; // Les streams dans l'espace de nom std
int main() {
   int i;
   cout << "Enter your favorite number:\n");
   cin >> i;
   cout << "Your favorite number is " << i << "\n";
   cerr << "An error message" << endl;
}
\end{lstlisting}

    \section*{Chaînes de caractère}

    \begin{lstlisting}
#include <string>

string a = "Hello", b = "World";
cout << a + b; // "Hello World"
cout << a + " You"; // "Hello You"
a.append(" Dog"); // En C++ les chaînes sont modifiables
cout << a; // "Hello Dog"
\end{lstlisting}

    \section*{Références}

    \begin{lstlisting}
string foo = "I am foo";
string& fooRef = foo; // Référence sur foo
fooRef += ". Hi!"; // Modifie foo via la référence
cout << fooRef; // Affiche "I am foo. Hi!"

cout << &fooRef << endl; // Affiche l'adresse de foo
string bar = "I am bar";
fooRef = bar;
cout << &fooRef << endl; // Affiche toujours l'adresse de foo
cout << fooRef; // Affiche "I am bar"

const string& barRef = bar;
barRef += ". Hi!"; // Erreur !

// Objets temporaires
string foo() { ... }
string retVal = foo(); // Valeur de retour est temporaire

void bar() {
  const string& constRef = foo();
  // Référence valide jusqu'à la fin de la fonction
}

string str;
void func(string& s) { ... } // Référence standard
void func(string&& s) { ... } // Référence temporaire
func(str); // Référence standard
func(foo()); // Référence temporaire

\end{lstlisting}

    \section*{Enumérations}

    \begin{lstlisting}
enum AnimalTypes {
    Mammals, Birds, Reptiles, Amphibians, Fish
};

AnimalTypes GetAnimalType() {
    return AnimalTypes::REPTILES;
}

enum Amphibians : uint8_t {
    Frogs, // 0
    Newts, // 1
    Salamanders = 42, // 42
    Toads // 43
}; // Peut être implicitement casté en uint8_t

enum class Birds : uint8_t {
    Chickens, Hummingbirds
}; // Ne sera pas implicitement casté
\end{lstlisting}

    \section*{Templates}

    \begin{lstlisting}
template<class T>
class Box {
public:
    void insert(const T&) { ... }
};

Box<int> intBox;
intBox.insert(42);

Box<Box<int> > boxOfBox;                // Template imbriquées
boxOfBox.insert(intBox);

template <class T> T f(T t);            // Surcharge f pour tous les types
template <class T> class X {            // Classe avec paramètre T
   X(T t); };                           // Constructeur
template <class T> X<T>::X(T t) {}      // Définition du constructeur
X<int> x(3);                            // Un objet de type X de int
template <class T, class U=T, int n=0>  // Template avec paramètres par défaults

// Paramètres
template<int Y>
void printMessage() {
   cout << "Your number is " << Y << endl;
}

// Spécialisation
template<>
void printMessage<42>() {
   cout << "You said 42!" << endl;
}
printMessage<23>(); // Your number is 23
printMessage<42>(); // You said 42!

\end{lstlisting}

    \section*{Namespaces}

    \begin{lstlisting}
extern "C" { void f(); }    // Si f() a été compilé dans un objet C
namespace N { class T {}; } // Cache la classe T dans le namespace N
N::T t;                     // Utilise le nom T du namespace N
using namespace N;          // Rends T visible sans préfix N::

using namespace std;        // Permet les appels standards sans std::

namespace First {
    namespace Nested {
        void foo() {
            printf("This is First::Nested::foo\n");
        }
    }
}
namespace Second {
    void foo() {
        printf("This is Second::foo\n");
    }
}
void foo() {
    printf("This is global foo\n");
}

int main() {
    using namespace Second;
    Second::foo();
    First::Nested::foo();
    ::foo(); // Global foo
    foo(); // Erreur car ambïgu
}
\end{lstlisting}

    \section*{Classes}

    \begin{lstlisting}
// Généralement déclaré dans un .h (ou .hpp)
class Dog {
    std::string name; // Privé par défaut (private:)
    int weight;
public:
    Dog(); // Constructeur par défaut

    // Constructeur avec liste d'initialisation
    Dog(std::string name, int weight=0) : name{name}, weight(weight)
    { ... }
    void setName(const std::string& dogsName);
    void setWeight(int dogsWeight);

    // Si marqué const, la méthode ne modifie pas l'état
    // de l'objet.
    virtual void print() const;

    // Méthode définie dans le body d'une classe
    void bark() const { std::cout << name << " barks!" << endl; }

    // Déstructeur doit être virtuel si peut être dérivé
    virtual ~Dog();

protected:
    // Accès possible aux classes dérivées

}

// Déclaration implémentée dans un .cpp
Dog::Dog() {
    std::cout << "A dog has been constructed\n";
}

Dog::~Dog() {
    std::cout << "Goodbye " << name << "\n";
}

int main() {
    Dog myDog; // Affiche "A dog has been constructed"
    myDog.setName("Barkley");
    myDog.setWeight(10);
} // Print "Goodbye Barkley"
\end{lstlisting}

    \section*{Héritage et polymorphisme}

    \begin{lstlisting}
class OwnedDog : public Dog {
public:
    void setOwner(const std::string& dogsOwner);
    void print() const override; // Surcharge
private:
    std::string owner;
};

void OwnedDog::print() const {
    Dog::print(); // Appel de la méthode du parent
    std::cout << "Dog is owned by " << owner << endl;
}
\end{lstlisting}

    \section*{Polymorphisme}

    \begin{lstlisting}
T operator+(T x, T y);  // Surcharge opérateur addition de type T
T operator-(T x);       // Surcharge opérateur unaire de negation -x
T operator++(int);      // Surcharge opérateur postfix

\end{lstlisting}

    \section*{Exceptions}

    \begin{lstlisting}
#include <exception>
#include <stdexcept>

try {
   throw std::runtime_error("A problem occured");
}
catch (const std::exception& ex) {
   std::count << ex.what();
}
catch (...) {
   std::cout << "Unknown exception caught";
   throw; // Relève l'exception
}
\end{lstlisting}

    \section*{Conteneurs de données}

    \subsection*{Vector}

    \begin{lstlisting}
#include <vector>

string val;
vector<string> vect;
cin >> val;
vect.push_back(val); // Ajoute à la fin
vect.pop_back(val);  // Enlève à la fin
vect.insert(iterator position, const value_type &val);
vect.erase(iterator position); // Efface une position
vect.erase(iterator first, iterator last); // Efface un intervalle

\end{lstlisting}

    \subsection*{Tuples (Liste non-modifiable)}

    \begin{lstlisting}
#include <tuple>

auto x = make_tuple(10, 'A');
const int maxN = 1e9;
const int maxL = 15;
auto y = make_tuple(maxN, maxL);

// Affichage
cout << get<0>(x) << " " << get<1>(x) << "\n"; // Affiche "10 A"
cout << get<0>(y) << " " << get<1>(y) << "\n"; // Affiche "1000000000 15"

// Unpacking
int i;
int c;
tie(i, c) = x;

tuple<int, char, double> z(11, 'A', 3.1415);

cout << tuple_size<decltype(z)>::value << "\n"; // Affiche 3

auto concatenated = tuple_cat(x, y, z); // (10, 'A', 1e9, 15, 11, 'A', 3.1415)

\end{lstlisting}

    \subsection*{Map (Association clé-valeur)}

    \begin{lstlisting}
#include <map>

map<char, int> mymap;
mymap.insert(pair<char,int>('A', 1));
mymap.insert(pair<char,int>('Z', 26));
map<char,int>::iterator it;

// Affiche A->1, Z->26
for (it=mymap.begin(); it!=mymap.end(); ++it)
    std::cout << it->first << "->" << it->second << ', ';

it = mymap.find('Z');
count << it->second; // Affiche 26

#include <unordered_map> // Plus performant mais ordre non préservé

\end{lstlisting}

    \subsection*{Set (Ensemble de valeurs uniques)}

    \begin{lstlisting}
#include <set>
set<int> ST;
ST.insert(4);
ST.insert(8);
ST.insert(15);
ST.erase(8);

set<int>::iterator it;
for(it=ST.begin();it<ST.end();it++) {
    cout << *it << endl;
}
ST.clear();
cout << ST.size(); // Affiche zéro

\end{lstlisting}

    \section*{Fonctions anonymes (lambda)}

    \begin{lstlisting}
vector<pair<int, int> > tester;
tester.push_back(make_pair(3,6));
tester.push_back(make_pair(1,9));
tester.push_back(make_pair(5,0));
sort(
    tester.begin(),
    tester.end(),
    [](const pair<int, int>& lhs, const pair<int, int>& rhs) {
        return lhs.second < rhs.second;
    }
);

struct S { void f(int i); };
void S::f(int i) {
    [&, i]{};      // Ok
    [&, &i]{};     // Erreur : i précédé par & quand & est le défaut
    [=, this]{};   // Erreur : this quand = est défaut
    [=, *this]{ }; // OK
    [i, i]{};      // Erreur : i répété
}
\end{lstlisting}

    \section*{Tokens alternatifs}

    \begin{lstlisting}
true and false          // Conjonction logique
true or false           // Disjonction logique
not true                // Négation logique

compl 4
4 bitor 3               // Disjonction bit à bit
4 bitand 3              // Conjonction bit à bit
4 xor 3                 // Disjonction exclusive bit à bit
\end{lstlisting}

    \section*{Smart Pointer}

    \begin{lstlisting}
void foo() {
   std::shared_ptr<Dog> doggo(new Dog());
   doggo->bark();
   // Pas besoin de supprimer le Dog.
}

\end{lstlisting}

    \section*{Étrangetés}

    \begin{lstlisting}
// Surcharge de méthode privées
class Foo {
private:
   virtual void bar();
};
class FooSub : public Foo {
   virtual void bar();      // Surcharge Foo::bar!
};

assert(0 == false and false == null); // La plupart du temps
\end{lstlisting}

    \section*{S.O.L.I.D.}
    \begin{description}
        \item[S] Single responsibility principle
        \item[O] Open for extension, closed for modification
        \item[L] Liskov substition principle
        \item[I] Interface segregation principle
        \item[D] Dependency inversion principle
    \end{description}

\end{multicols*}
\end{document}
